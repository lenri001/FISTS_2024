% Theorem Styles

\theoremstyle{theorem}
\newtheorem{thm}{Theorem}
\newtheorem{lm}{Lemma}
\newtheorem{ex}{Example}
\newtheorem{exs}{Examples}

\theoremstyle{definition}
\newtheorem{defn}{Definition}
\newtheorem{notation}{Notation}

\theoremstyle{remark}
\newtheorem*{remark}{Remark}

% Commands

\DeclareMathOperator{\diam}{diam}
\DeclareMathOperator{\id}{id}

\newcommand{\8}{\infty}
\newcommand{\A}{\mathscr{A}}
\newcommand{\C}{\mathbb{C}}
\newcommand{\N}{\mathbb{N}}
\newcommand{\Q}{\mathbb{Q}}
\newcommand{\R}{\mathbb{R}}
\newcommand{\Z}{\mathbb{Z}}
\newcommand{\iunit}{\mathbbm{i}}

\newcommand{\dif}{\mathrm{d}}
\newcommand{\diff}{\,\mathrm{d}}
\newcommand{\Dim}[1]{\dim_{\mathrm{#1}}}
\newcommand{\st}{\,|\,}
\newcommand{\ST}{\,\middle|\,}
\newcommand{\Poles}{\mathscr{P}}
\renewcommand\vec[1]{\boldsymbol{#1}}
\newcommand\Union{\mathop{\bigcup}}
\newcommand\Intersect{\mathop{\bigcap}}
\newcommand\union{\mathrel{\cup}}
\newcommand\intersect{\mathrel{\cap}}

\newgray{darkgrey}{.25}
\newgray{grey}{.5}
\newgray{lightgrey}{.75}
\newgray{vlightgrey}{.9}

\newcommand{\secslide}[2]{
	\section[#1]{#2}
		\begin{frame}
			\begin{center}
				\ifdefined\ishandout
					\textbf{\Huge\secname}
				\else
					\textcolor{structure}{\textbf{\Huge\secname}}
				\fi
			\end{center}
		\end{frame}
}

\renewcommand{\baselinestretch}{1.1}
