%%%%%%%%%%%%%%%%%%%%%%%%%%%%%%%%%%%%%%%%%
% Jacobs Landscape Poster
% LaTeX Template
% Version 1.0 (29/03/13)
%
% Created by:
% Computational Physics and Biophysics Group, Jacobs University
% https://teamwork.jacobs-university.de:8443/confluence/display/CoPandBiG/LaTeX+Poster
% 
% Further modified by:
% Nathaniel Johnston (nathaniel@njohnston.ca)
%
% This template has been downloaded from:
% http://www.LaTeXTemplates.com
%
% License:
% CC BY-NC-SA 3.0 (http://creativecommons.org/licenses/by-nc-sa/3.0/)
%
%%%%%%%%%%%%%%%%%%%%%%%%%%%%%%%%%%%%%%%%%

%----------------------------------------------------------------------------------------
%	PACKAGES AND OTHER DOCUMENT CONFIGURATIONS
%----------------------------------------------------------------------------------------

\documentclass[final, 20 pt]{beamer}

\usepackage[size = a0, scale=1.2]{beamerposter} % Use the beamerposter package for laying out the poster
\usepackage{pdfpages}
\usepackage{amsmath,amsthm,amssymb}
\usepackage{tikz}
\usepackage{pstricks}
\usepackage{ragged2e}
\usepackage{mathrsfs}
\usepackage{microtype}
\usepackage{bbm}
\usepackage{graphicx, wrapfig, subcaption, setspace, booktabs}
\usepackage{tabularx}
\usepackage{listings}
\usepackage{xcolor}
\usepackage{longtable}
%\usepackage{titlesec}
\usepackage{url, lipsum}
\usepackage{hyperref,bookmark}
\usepackage[T1]{fontenc}
\usepackage{amssymb}
\usepackage{orcidlink}
\usepackage{tabularx}
\usepackage{listings}
\usepackage{xcolor}
\usepackage{longtable}
\usepackage{setspace}
\usepackage{float}
\usepackage{multirow}
\usepackage[ruled,linesnumbered]{algorithm2e}


\usetheme{confposter} % Use the confposter theme supplied with this template

\setbeamercolor{block title}{fg=ngreen,bg=white} % Colors of the block titles
\setbeamercolor{block body}{fg=black,bg=white} % Colors of the body of blocks
\setbeamercolor{block alerted title}{fg=white,bg=dblue!70} % Colors of the highlighted block titles
\setbeamercolor{block alerted body}{fg=black,bg=dblue!10} % Colors of the body of highlighted blocks
% Many more colors are available for use in beamerthemeconfposter.sty

%-----------------------------------------------------------
% Define the column widths and overall poster size
% To set effective sepwid, onecolwid and twocolwid values, first choose how many columns you want and how much separation you want between columns
% In this template, the separation width chosen is 0.024 of the paper width and a 4-column layout
% onecolwid should therefore be (1-(# of columns+1)*sepwid)/# of columns e.g. (1-(4+1)*0.024)/4 = 0.22
% Set twocolwid to be (2*onecolwid)+sepwid = 0.464
% Set threecolwid to be (3*onecolwid)+2*sepwid = 0.708

\newlength{\sepwid}
\newlength{\onecolwid}
\newlength{\twocolwid}
\newlength{\threecolwid}
\setlength{\paperwidth}{48in} % A0 width: 46.8in
\setlength{\paperheight}{36in} % A0 height: 33.1in
\setlength{\sepwid}{0.024\paperwidth} % Separation width (white space) between columns
\setlength{\onecolwid}{0.22\paperwidth} % Width of one column
\setlength{\twocolwid}{0.464\paperwidth} % Width of two columns
\setlength{\threecolwid}{0.708\paperwidth} % Width of three columns
\setlength{\topmargin}{-0.5in} % Reduce the top margin size
%-----------------------------------------------------------

\usepackage{graphicx}  % Required for including images

\usepackage{booktabs} % Top and bottom rules for tables

%----------------------------------------------------------------------------------------
%	TITLE SECTION 
%----------------------------------------------------------------------------------------

\title{Microgrid Demand Response: A Comparison of Simulated and Real Results} % Poster title

\author{Luis Fernando Enriquez-Contreras, A S M Jahid Hasan,  Jubair Yusuf, Jacqueline Garrido, Sadrul Ula} % Author(s)

\institute{Department of Electrical and Computer Engineering  \\ University of California, Riverside} % Institution(s)

%----------------------------------------------------------------------------------------

\begin{document}

\addtobeamertemplate{block end}{}{\vspace*{2ex}} % White space under blocks
\addtobeamertemplate{block alerted end}{}{\vspace*{2ex}} % White space under highlighted (alert) blocks

\setlength{\belowcaptionskip}{2ex} % White space under figures
\setlength\belowdisplayshortskip{2ex} % White space under equations

\begin{frame}[t] % The whole poster is enclosed in one beamer frame

\begin{columns}[t] % The whole poster consists of three major columns, the second of which is split into two columns twice - the [t] option aligns each column's content to the top

\begin{column}{\sepwid}\end{column} % Empty spacer column

\begin{column}{\onecolwid} % The first column

%----------------------------------------------------------------------------------------
%	OBJECTIVES
%----------------------------------------------------------------------------------------

\begin{alertblock}{Purpose}

\begin{itemize}
	\item The goal is to see the accuracy of simulated microgrid models when applied to a real system
	\item This validation's main contribution is documenting the difficulties and challenges involved in transitioning from a simulated model to real-life implementation
	\item This paper focuses on the problems and outcomes of the microgrid's full-size components and describes how software is implemented in a full-scale microgrid
	\item The validation results are compared to the simulated model to analyze the differences and similarities between simulated and real-life data
\end{itemize}

\end{alertblock}

%----------------------------------------------------------------------------------------
%	INTRODUCTION
%----------------------------------------------------------------------------------------

\begin{block}{Abstract}
	Microgrids can be used in demand response (DR) and islanding operations. This paper explores a real-world microgrid implementation performed at the University of California, Riverside. The intended purpose is to see the feasibility of demand response and islanding in a real microgrid. An algorithm is used to generate ``target data" for the Battery Energy Storage (BES) Inverter to follow. The other purpose is to see the accuracy when transitioning from a simulated model to a real-life implementation. The results from this implementation show it is feasible to respond to demand response scenarios with the current setup. However, more research is needed to successfully implement islanding responses due to infrequent power spikes that briefly draw more power than the microgrid can produce.

\end{block}

%------------------------------------------------

%\begin{figure}
%\includegraphics[width=0.8\linewidth]{placeholder.jpg}
%\caption{Figure caption}
%\end{figure}

%----------------------------------------------------------------------------------------

\end{column} % End of the first column

\begin{column}{\sepwid}\end{column} % Empty spacer column

\begin{column}{\twocolwid} % Begin a column which is two columns wide (column 2)

\begin{columns}[t,totalwidth=\twocolwid] % Split up the two columns wide column

\begin{column}{\onecolwid}\vspace{-.6in} % The first column within column 2 (column 2.1)

%----------------------------------------------------------------------------------------
%	Setup
%----------------------------------------------------------------------------------------

\begin{block}{Setup}
	\begin{figure}[!htb] 		
		\includegraphics[width=\linewidth,keepaspectratio, angle=0]{Fig/power_system_setup.pdf}
		\caption{Power System Setup for Office Building Microgrid}
		\label{pow}			
	\end{figure}
\end{block}

%----------------------------------------------------------------------------------------

\end{column} % End of column 2.1

\begin{column}{\onecolwid}\vspace{-.6in} % The second column within column 2 (column 2.2)

%----------------------------------------------------------------------------------------
%	METHODS
%----------------------------------------------------------------------------------------

\begin{block}{Control Software}

\begin{algorithm}[H]
	\caption{Microgrid Control Software}\label{alg:control_software}
	df $\gets$ dataset \\
	output $\gets$ [] \\
	count $\gets$ 0 \\
	delay $\gets$ 15 minutes \\
	new\_interval $\gets$ current\_time \\
	\While {data\_length $>$ count}
	{
		inverter\_power $\gets$ df.power[count]\\
		\While{new\_interval $>$ current\_time}
		{
			output $\gets$ inverter\_get\_data
		}
		count += 1 \\
		new\_interval += delay \\
	}	
\end{algorithm}

\end{block}

%----------------------------------------------------------------------------------------

\end{column} % End of column 2.2

\end{columns} % End of the split of column 2 - any content after this will now take up 2 columns width

%----------------------------------------------------------------------------------------
%	IMPORTANT RESULT
%----------------------------------------------------------------------------------------

%\begin{alertblock}{Important Result}
%
%Lorem ipsum dolor \textbf{sit amet}, consectetur adipiscing elit. Sed commodo molestie porta. Sed ultrices scelerisque sapien ac commodo. Donec ut volutpat elit.
%
%\end{alertblock} 

%----------------------------------------------------------------------------------------

\begin{columns}[t,totalwidth=\twocolwid] % Split up the two columns wide column again

\begin{column}{\onecolwid} % The first column within column 2 (column 2.1)

%----------------------------------------------------------------------------------------
%	MATHEMATICAL SECTION
%----------------------------------------------------------------------------------------

\begin{block}{Optimization Algorithim}
\begin{algorithm}[H]
	\caption{Optimize Net Load}\label{alg:opt}
	Read solar values $P^{S}$ for electric meter\\
	Get net load $P^{G}$ \\
	Convert to 15 minute data \\ 
	Add $P^{G}$ and $P_{S}$  to get $P^{L}$\\
	$P^{B}_{t} \geq 0$ = Charging \\
	$P^{B}_{t} < 0$ = Discharging \\
	Optimize $P^{G}$ for $P^{B}$, $P^{S}$, and  $P^{L}$\\
	%			\scriptsize
	\begin{equation} \label{objective_2} 
	\min_{P^{G}, P^{B}, P^{S}, P^{L}} \ f(P^{G})
	\end{equation}
	\normalsize
	subject to
	\begin{equation} \label{processing_power}
	E^{B}_{t + 1} = E^{B}_{t} + P^{B}_{t} \Delta t
	\end{equation}
	\begin{equation} \label{server_capacity}
	E^{B_{min}} \leq  E^{B}_{t} \leq E^{B_{max}}
	\end{equation}
	\begin{equation} \label{server_size}
	-100 \leq P^{B} \leq 100
	\end{equation}
	\begin{equation} \label{flow_size}
	P^{L} = P^{G} + P^{S} + P^{B}
	\end{equation}
	%			Use object function min \\
	%			Charging Command \\
\end{algorithm}
$P^{G}$ = Power from/sent to Grid, $P^{B}$ Power from/consumed by the battery, $P^{S}$ Power from Solar, $P^{L}$ Power consumed by Building, and Energy of the Battery (State of Charge)

\end{block}

%----------------------------------------------------------------------------------------

\end{column} % End of column 2.1

\begin{column}{\onecolwid} % The second column within column 2 (column 2.2)

%----------------------------------------------------------------------------------------
%	RESULTS
%----------------------------------------------------------------------------------------

\begin{block}{Results}

	\begin{figure}[!htb]
		\centering
		\begin{subfigure}[b]{0.475\textwidth}
			\centering
			\includegraphics[width=\textwidth]{Fig/ucr_theortical_actual_net_load_11_20.pdf}
			\caption[Network2]%
			{{\small UCR Microgrid Building Theoretical and Actual Load Data }}    
			\label{ucr_act}
		\end{subfigure}
		\hfill
		\begin{subfigure}[b]{0.475\textwidth}  
			\centering 
			\includegraphics[width=\textwidth]{Fig/ucr_theortical_actual_net_load_resample_11_20.pdf}
			\caption[]%
			{{\small UCR Microgrid Building Theoretical and Actual Load 15 Minute Average Resample}}    
			\label{ucr_act_rs}
		\end{subfigure}
		\vskip\baselineskip
		
		\caption[ The average and standard deviation of critical parameters ]
		{\small UCR Microgrid Building Theoretical and Actual Load} 
		\label{fig:ucr_act}
	\end{figure}

	\begin{figure}[!htb] 		
		\includegraphics[width=\linewidth,keepaspectratio, angle=0]{Fig/ucr_ivt_bl_sol_power_11_20.pdf}
		\caption{UCR Microgrid Inverter, Solar, and Building Power}
		\label{mg_pow}			
	\end{figure}

\end{block}

%----------------------------------------------------------------------------------------

\end{column} % End of column 2.2

\end{columns} % End of the split of column 2

\end{column} % End of the second column

\begin{column}{\sepwid}\end{column} % Empty spacer column

\begin{column}{\onecolwid} % The third column

%----------------------------------------------------------------------------------------
%	CONCLUSION
%----------------------------------------------------------------------------------------

\begin{block}{Conclusion}

	\begin{itemize}	
		\item The results from this short validation demonstrate the possibility lowering the max demand of the net load, thus reducing the facility’s demand costs, by utilizing a BES system in combination with solar power
		\item The algorithm proposed and implemented achieved the goal of lowering the max demand, thereby reducing the stress on the grid
		\item Load prediction is needed to optimize data for the algorithm more accurately; optimization using shorter time intervals is also needed to reduce possible power surges
		\item This approach is unsuitable for islanding since the net load would exceed zero on multiple occasions
		\item Future improvements on this microgrid system will integrate load control, and a faster reaction time to the system that should enable full capability of automated islanding
	\end{itemize}

\end{block}

%----------------------------------------------------------------------------------------
%	ADDITIONAL INFORMATION
%----------------------------------------------------------------------------------------

%\begin{block}{Additional Information}
%
%Maecenas ultricies feugiat velit non mattis. Fusce tempus arcu id ligula varius dictum. 
%\begin{itemize}
%\item Curabitur pellentesque dignissim
%\item Eu facilisis est tempus quis
%\item Duis porta consequat lorem
%\end{itemize}
%
%\end{block}

%----------------------------------------------------------------------------------------
%	REFERENCES
%----------------------------------------------------------------------------------------

%\begin{block}{References}
%
%\nocite{*} % Insert publications even if they are not cited in the poster
%\small{\bibliographystyle{unsrt}
%\bibliography{sample}\vspace{0.75in}}
%
%\end{block}

%----------------------------------------------------------------------------------------
%	ACKNOWLEDGEMENTS
%----------------------------------------------------------------------------------------

\setbeamercolor{block title}{fg=red,bg=white} % Change the block title color

\begin{block}{Acknowledgements}

I would like to give a special thanks to Henry Gomez, the CE-CERT staff, and my family who have supported me throughout my research and without them this contribution would not be possible. \\

\end{block}

%----------------------------------------------------------------------------------------
%	CONTACT INFORMATION
%----------------------------------------------------------------------------------------

\setbeamercolor{block alerted title}{fg=black,bg=norange} % Change the alert block title colors
\setbeamercolor{block alerted body}{fg=black,bg=white} % Change the alert block body colors

\begin{alertblock}{Contact Information}

\begin{itemize}
\item Researcher: Luis Fernando Enriquez-Contreras
\item Web: \href{https://www.cert.ucr.edu/renewable-energy-production-integration/sustainable-integrated-grid-initiative-sigi}{https://www.cert.ucr.edu/renewable-energy-production-integration/sustainable-integrated-grid-initiative-sigi}
\item Email: \href{lenri001@ucr.edu}{lenri001@ucr.edu}
\item Phone: +1 (909) 763 1899
\end{itemize}

\end{alertblock}

\begin{center}
\begin{tabular}{ccc}
%\includegraphics[width=0.4\linewidth]{logo.png} & \hfill & %\includegraphics[width=0.4\linewidth]{logo.png}
\end{tabular}
\end{center}

%----------------------------------------------------------------------------------------

\end{column} % End of the third column

\end{columns} % End of all the columns in the poster

\end{frame} % End of the enclosing frame

\end{document}