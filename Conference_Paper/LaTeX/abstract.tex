\documentclass[conference]{IEEEtran}
\IEEEoverridecommandlockouts
% The preceding line is only needed to identify funding in the first footnote. If that is unneeded, please comment it out.
\usepackage{cite}
\usepackage{algorithmic}
\usepackage{graphicx}
\usepackage{textcomp}
\usepackage{amsmath}
\usepackage[sc]{mathpazo}
\usepackage{datetime}
\usepackage{graphicx, wrapfig, subcaption, setspace, booktabs}
\usepackage[T1]{fontenc}
\usepackage{fourier}
\usepackage{url, lipsum}
\usepackage{hyperref,bookmark}
\usepackage[T1]{fontenc}
\usepackage{amssymb}
\usepackage{makecell, multirow, tabularx}
\usepackage{listings}
\usepackage[ruled,linesnumbered,vlined]{algorithm2e}

\usepackage{float}
\usepackage{nomencl}
\usepackage{ifthen}

\renewcommand{\nomgroup}[1]{%
	\ifthenelse{\equal{#1}{P}}{\item[\textbf{Parameters}]}{%
		\ifthenelse{\equal{#1}{V}}{\item[\textbf{Variables \\}]}}
}
\makenomenclature
\def\BibTeX{{\rm B\kern-.05em{\sc i\kern-.025em b}\kern-.08em
    T\kern-.1667em\lower.7ex\hbox{E}\kern-.125emX}}
\begin{document}

\title{CO\textsubscript{2} and Cost Impacts of a Microgrid  with Electric Vehicle \\Charging Infrastructure: a Case Study in Southern California}

\author{
	\IEEEauthorblockN{Luis Fernando Enriquez-Contreras\IEEEauthorrefmark{1}\IEEEauthorrefmark{2}, Matthew Barth\IEEEauthorrefmark{1}\IEEEauthorrefmark{2}, Sadrul Ula\IEEEauthorrefmark{2}}
	\IEEEauthorblockA{\IEEEauthorrefmark{1}\textit{Department of Electrical  and Computer Engineering} \\
		\textit{University of California, Riverside}\\
				Riverside, United States of America \\
				lenri001@ucr.edu, barth@ece.ucr.edu}
	\IEEEauthorblockA{\IEEEauthorrefmark{2}\textit{College of Engineering, Center for Environmental Research \& Technology} \\
		\textit{University of California, Riverside}\\
				Riverside, United States of America \\
				lenri001@ucr.edu, barth@ece.ucr.edu, sula@cert.ucr.edu}
}
\maketitle

\begin{abstract}
	As an important part of Intelligent Transportation Systems (ITS), this paper presents a case study at the University of California, Riverside (UCR) that evaluates the effectiveness of different transportation-based microgrid configurations in reducing both carbon dioxide (CO\textsubscript{2}) emissions and electricity costs. CO\textsubscript{2} emissions are calculated using high-resolution California Independent System Operator (CAISO) CO\textsubscript{2} emissions data to accurately assess the environmental impact of each setup. Electric costs were also compared to determine the financial savings potential for the consumer. The results demonstrate that a peak-shaving transportation-microgrid strategy can effectively reduce CO\textsubscript{2} emissions in the range of 24\% to 38\% and costs from \$27,000 to \$29,000 per year, even when considering the additional demand from 12 vehicles charging daily at the building. However, careful consideration should be given to battery sizing, as peak-shaving has diminishing returns. Doubling the battery size may only provide an additional savings of \$2,000 per year with a negligible reduction in emissions. This highlights the importance of optimizing battery capacity to maximize cost-effectiveness and environmental impact.
\end{abstract}
\begin{IEEEkeywords}
microgrids, demand response, CO\textsubscript{2} emissions, modelica, EV charging
\end{IEEEkeywords}

		
\end{document}
